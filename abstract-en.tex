The scientific community relies more and more on computations, notably for numerical simulations and data processing.
While many scientific advances were made possible by the technological progress of computers, additional performance
gains are still required for larger scale projects.

The race for performance is obtained with a growing hardware and software complexity, which in turn increases the
performance variability. This can make the experimental study of performance extremely challenging, raising concerns of
reproducibility of the experiments, akin to the problems already faced by natural sciences.

Our contributions are twofold. First, we present a methodology for \emph{predicting} the performance of parallel
non-trivial applications through simulation. We describe several models for the communications and the computations,
with an increased complexity. We compare these models through an extensive validation by matching our predictions with
real experiments. This validation shows that modeling the spatial and temporal variability of the platform is essential
for faithful predictions, the effect of which is then quantified through several sensibility analyses.  Second, we
present the lessons learned while making the numerous experiments required in the first part and how we improved our
methodology. We show how we can suffer from multiple experimental bias while \emph{measuring} performance and how we can
overcome this. We also present our implementation of performance non-regression \emph{testing}, which allowed us to
detect many significant changes of the platform throughout this thesis.
