The scientific community relies more and more on computations, notably for numerical simulations and data processing.
While many scientific advances were made possible by the technological progress of computers, additional performance
gains are still required for larger scale projects.

The race for performance is addressed with a growing hardware and software complexity, which in turn increases the
performance variability. This can make the experimental study of performance extremely challenging, raising concerns of
reproducibility of the experiments, akin to the problems already faced by natural sciences.

Our contributions are twofold. First, we present a methodology for predicting the performance of parallel non-trivial
applications through simulation. We describe several models for communications and computations, with an increasing
complexity. We compare these models through an extensive validation by matching our predictions with real experiments.
This validation shows that modeling the spatial and temporal variability of the platform is essential for faithful
predictions. As a consequence, predictions require careful sensibility analysis accounting for the uncertainty on the
resource models, which we illustrate through several case studies.  Second, we present the lessons learned while making
the numerous experiments required in the first part and how we improved our methodology. We show that measurements can
suffer from multiple experimental bias and we explain how some of these bias can be overcome. We also present how we
implemented systematic performance non-regression testing, which allowed us to detect many significant changes of the
platform throughout this thesis.
