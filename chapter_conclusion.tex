\chapter{Discussion}
\label{chapter:conclusion}

    %% TODO
    %% - SimGrid suppose que les consommations réseau et cpu sont linéaires
    %%   et indépendantes. Ce n'est pas si simple. Tension entre
    %%   mesurer/injecter et modéliser/prédire.
    %% - Ça marche mais niveau d'expertise encore assez démentiel
    %% - Résolution/automatisation de pas mal de choses en terme de
    %%   modélisation de plate-forme mais il y a encore beaucoup de
    %%   travail. LibSimBLAS ?

    \section{Contribution}%

        This thesis has contributed to the improvement of experimental reproducibility in high performance computing.

        In Part~\ref{part:prediction}, we described a method for predicting the performance of MPI applications through
        simulation. Using Simgrid/SMPI simulator, we managed to emulate the High Performance Linpack benchmark at scale
        by applying only a few modifications to its source code. We compared several computation and communication
        models and showed the importance of modeling both the temporal and spatial variability of the platform. In a
        thorough comparison of the simulations with real executions, we show that the prediction error remains very low,
        only a few percent, thereby demonstrating the faithfulness of this approach. Several sensibility analyses are
        then performed to quantify the effect of platform variability and showcase an important use case of simulation.

        The lessons learned during this work are then presented in Part~\ref{part:experiment}. We start by describing the
        experiment engine we developed and that was used throughout this thesis. Then, we present an in-depth report of
        the many experimental biases we faced, including very unsettling phenomenons we did not anticipate. While some
        of these biases can be desirable if they are also occurring in the simulated application, most of them had to be
        suppressed through randomization. Finally, we showcase the performance non-regression test we implemented. While
        not a statistical novelty, they allowed us to detect many changes on Grid'5000 platform that affected
        significantly the performance and could harm experiments if gone unnoticed. We believe the HPC community could
        greatly benefit of such tests.

    \section{Future work}%

        Besides the next steps already discussed in Chapter~\ref{chapter:prediction:conclusion} and
        Section~\ref{sec:test:conclusion}, there also remains a unification work. By going a step further in the
        automatization, we could use the data produced by the non-regression tests to generate new model instances for
        Simgrid. It would then be possible to make new simulations with a platform model that reflects the latest
        changes of the real platform. Then, by implementing the same statistical test on the performance predicted by
        the simulation, we should be able to detect whether a platform change has affected the application
        performance and quantify this effect. A minor performance drop of the computation kernels could be amplified by
        the synchronization phases of the application and become very concerning at a larger scale. Conversely, it could
        also be attenuated by the global noise and go completely unnoticed.
