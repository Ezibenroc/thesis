\chapter{Experimental control}
\label{chapter:experiment}

\section{Experimental Testbed and Experiment Engines}%
\label{sec:experiment:testbed}

    \subsection{State of the art}%
    \label{sub:state_of_the_art}

        \subsubsection{Grid'5000}%
        \label{ssub:grid_5000}

            Nearly all the experiments presented in this document have been carried on the Grid'5000~\cite{grid5000}
            testbed.  Quoting its official website\footnote{\url{https://www.grid5000.fr/}}: \blockquote{Grid'5000 is a
            large-scale and flexible testbed for experiment-driven research in all areas of computer science, with a
            focus on parallel and distributed computing including Cloud, HPC and Big Data and AI.} It provides dozen of
            clusters, each one having between 2 and 124 homogeneous compute nodes. There is a high diversity of hardware,
            including several generations of Intel processors available, AMD and ARM processors, GPU, persistent memory
            (PMEM) as well as high-performance networks such as Infiniband or Omni-path. Another important feature is the
            ability for the experimenter to get full control on the nodes, as it is possible to deploy a new operating
            system and therefore to gain superuser access.

        \subsubsection{Experiment engines}%
        \label{ssub:experiment_engines}




\section{Randomizing matters!}%
\label{sec:experiment:randomizing}

    some text...

\section{Performance non-regression tests}%
\label{sec:experiment:tests}

    some text...
