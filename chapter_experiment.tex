\part{Experimental control}
\label{part:experiment}

\chapter{Experimental Testbed and Experiment Engines}%
\label{chapter:experiment:testbed}

    \section{State of the art}%
    \label{sec:state_of_the_art}

        \subsection{Grid'5000}%
        \label{sub:grid_5000}

            Nearly all the experiments presented in this document have been carried on the Grid'5000~\cite{grid5000}
            testbed.  Quoting its official website\footnote{\url{https://www.grid5000.fr/}}: \blockquote{Grid'5000 is a
            large-scale and flexible testbed for experiment-driven research in all areas of computer science, with a
            focus on parallel and distributed computing including Cloud, HPC and Big Data and AI.} It provides dozen of
            clusters, each one having between 2 and 124 homogeneous compute nodes. There is a high diversity of hardware,
            including several generations of Intel processors available, AMD and ARM processors, GPU, persistent memory
            (PMEM) as well as high-performance networks such as Infiniband or Omni-path. Another important feature is the
            ability for the experimenter to get full control on the nodes, as it is possible to deploy a new operating
            system and therefore to gain superuser access.

        \subsection{Experiment engines}%
        \label{sub:experiment_engines}

            While it is possible to run a complete experiment on a testbed like Grid'5000 by manually issuing commands
            in an interactive shell, it is not advisable as it quickly becomes extremly tedious and error-prone.
            Automating the experiment is a necessary condition to have reproducible results. A first step toward this
            goal is to write some ad-hoc script. However, two independent experiments might still share a lot of steps
            that could be refactored in a common layer, \eg OS deployment, package installation, or even more advanced
            features like node instrumentation or environment logging.

            For these reasons, it is a common practice to use an experiment engine. Buchert~\etal describe the features
            of eight different softwares~\cite{buchert:hal-01087519}. To the best of our knowledge, only three offer a
            native support for Grid'5000, namely Expo, XPFlow and Execo. Unfortunately, Expo and XPFlow are now longer
            maintained, the last commit in their respective repositories was done on November 2014 and September 2015
            For these reasons, the experiment engine Execo~\cite{Imbert_2013} is often recommended to Grid'5000
            newcomers.

\chapter{Randomizing matters!}%
\label{chapter:experiment:randomizing}

    some text...

\chapter{Performance non-regression tests}%
\label{chapter:experiment:tests}

    some text...
