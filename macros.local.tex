% *** SPECIFIC MACROS ***
% specific macro definition for the paper
%
% requires amsthm, xspace

% theorem-like environment
\newtheorem{thm}{Theorem}
\newtheorem{lem}[thm]{Lemma}
\newtheorem{prop}{Proposition}
\newtheorem{propty}{Property}
\newtheorem{defn}{Definition}

% asymptotic complexity (Landau) notations
\newcommand{\landauO}{\ensuremath{\mathcal{O}}\xspace}
\newcommand{\landauOmega}{\ensuremath{\Omega}\xspace}
\newcommand{\landauTheta}{\ensuremath{\Theta}\xspace}
\newcommand{\landauo}{\ensuremath{o}\xspace}
\newcommand{\landauomega}{\ensuremath{\omega}\xspace}
\newcommand{\landauorder}{\ensuremath{\sim}\xspace}

% scheduling notations
\newcommand{\graham}[3]{\mbox{\ensuremath{#1\mid#2\mid#3}}}
\newcommand{\Cmax}{\ensuremath{C_{\max}}\xspace}

% complexity classes
\newcommand{\cNP}{\textbf{NP}\xspace}  % bold notations
\newcommand{\cP}{\textbf{P}\xspace}
% \newcommand{\cNP}{\ensuremath{\mathcal{N\!P}}\xspace}  % round notations
% \newcommand{\cP}{\ensuremath{\mathcal{P}}\xspace}

% Machine states
\newcommand{\computing}{computing}
\newcommand{\idle}{idle}
\newcommand{\off}{of\!f}
\newcommand{\on}{on}
\newcommand{\ontooff}{\on\rightarrow\off}
\newcommand{\offtoon}{\off\rightarrow\on}

% CCGRID 2017
\newcommand{\ra}[1]{\renewcommand{\arraystretch}{#1}}

% Cluster 2017
\newcommand{\overbar}[1]{\mkern 1.5mu\overline{\mkern-1.5mu#1\mkern-1.5mu}\mkern 1.5mu}

\newlength\mylen % algorithm2e hack
\newcommand\myinput[1]{%
  \settowidth\mylen{\KwIn{}}%
  \setlength\hangindent{\mylen}%
  \hspace*{\mylen}#1\\}

\newcommand\myoutput[1]{% algorithm2e hack
  \settowidth\mylen{\KwOut{}}%
  \setlength\hangindent{\mylen}%
  \hspace*{\mylen}#1\\}



%%%%%%%%%%%%%%%%%%%%%%%
% Modular compilation %
%%%%%%%%%%%%%%%%%%%%%%%
\newif\ifwatermark
\watermarktrue

\newif\iftotalcompilation
\totalcompilationtrue

\newcommand{\inputchapter}[2]{%
  \ifdef{#1}%
    {\input{#2}\cleardoublepage}%
    {}%
}

% Modeling notations
\newcommand{\model}[2][]{\ensuremath{\mathcal{M}_{#1}\ifthenelse{\equal{#2}{}}{}{\!-\!}{#2}}\xspace}
\newcommand{\modelp}[2][]{\ensuremath{\mathcal{M'}_{#1}\!\ifthenelse{\equal{#2}{}}{}{\!-\!}{#2}}\xspace}
\newcommand{\noise}[2][]{\ensuremath{\mathcal{N}_{#1}\ifthenelse{\equal{#2}{}}{}{\!-\!}{#2}}\xspace}
\newcommand{\noisep}[2][]{\ensuremath{\mathcal{N'}_{#1}\!\ifthenelse{\equal{#2}{}}{}{\!-\!}{#2}}\xspace}
\newcommand{\norm}{\ensuremath{N}\xspace}
\newcommand{\mcdots}{\ensuremath{\!\cdot\!\cdot\!\cdot\!}\xspace}
\newcommand{\unif}[2]{\ensuremath{\mathcal{U}\left({#1},{#2}\right)}\xspace}

% Abreviations
\newcommand{\eg}{e.g.\@\xspace}
\newcommand{\ie}{i.e.\@\xspace}
\newcommand{\resp}{resp.\@\xspace}
\newcommand{\etal}{et~al.\@\xspace}
\newcommand{\dgemm}{\texttt{dgemm}\@\xspace}
\newcommand{\recv}{\texttt{MPI\_Recv}\@\xspace}
\newcommand{\send}{\texttt{MPI\_Send}\@\xspace}

% Referencing several times a footnote
% See https://tex.stackexchange.com/a/54240
\newcommand{\savefootnote}[2]{\footnote{\label{#1}#2}}
\newcommand{\repeatfootnote}[1]{\textsuperscript{\ref{#1}}}
