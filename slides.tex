\documentclass[10pt]{beamer}

\usepackage[ruled,vlined,english]{algorithm2e}
\DontPrintSemicolon

\usepackage{graphicx}
\usepackage[normalem]{ulem}
\usepackage{color,soul}
\usepackage{multirow}
\usepackage{tikz}
\usetikzlibrary{decorations.pathreplacing}
\usetikzlibrary{calc}
\usetikzlibrary{positioning}
\usetikzlibrary{arrows.meta}
\usetikzlibrary{shapes}
\usetheme[numbering=fraction,titleformat=smallcaps,progressbar=frametitle]{metropolis}
\usepackage[normalem]{ulem}

\usepackage{svg}

\usepackage[binary-units,group-digits,group-separator={,}]{siunitx}
\DeclareSIUnit\flop{Flop}
\DeclareSIUnit\flops{\flop\per\second}
\newcommand{\Num}[1]{\num[group-separator={,}]{#1}\xspace}
\newcommand{\NSI}[2]{\SI[group-separator={,}]{#1}{#2}\xspace}


% From https://github.com/matze/mtheme/issues/237#issuecomment-258718692
\makeatletter
\setlength{\metropolis@titleseparator@linewidth}{1.5pt}
\setlength{\metropolis@progressonsectionpage@linewidth}{1.5pt}
\setlength{\metropolis@progressinheadfoot@linewidth}{1pt}
\makeatother

\usepackage{fontawesome}

\newcommand{\backupbegin}{
   \newcounter{framenumberappendix}
   \setcounter{framenumberappendix}{\value{framenumber}}
}
\newcommand{\backupend}{
   \addtocounter{framenumberappendix}{-\value{framenumber}}
   \addtocounter{framenumber}{\value{framenumberappendix}}
}

% Command to make the last slide, which shows copies of the other slides
\newcommand{\addframe}[1] {\includegraphics[page=#1, width=0.3\textwidth]{demo_old.pdf}}

% *** SPECIFIC MACROS ***
% specific macro definition for the paper
%
% requires amsthm, xspace

% theorem-like environment
\newtheorem{thm}{Theorem}
\newtheorem{lem}[thm]{Lemma}
\newtheorem{prop}{Proposition}
\newtheorem{propty}{Property}
\newtheorem{defn}{Definition}

% asymptotic complexity (Landau) notations
\newcommand{\landauO}{\ensuremath{\mathcal{O}}\xspace}
\newcommand{\landauOmega}{\ensuremath{\Omega}\xspace}
\newcommand{\landauTheta}{\ensuremath{\Theta}\xspace}
\newcommand{\landauo}{\ensuremath{o}\xspace}
\newcommand{\landauomega}{\ensuremath{\omega}\xspace}
\newcommand{\landauorder}{\ensuremath{\sim}\xspace}

% scheduling notations
\newcommand{\graham}[3]{\mbox{\ensuremath{#1\mid#2\mid#3}}}
\newcommand{\Cmax}{\ensuremath{C_{\max}}\xspace}

% complexity classes
\newcommand{\cNP}{\textbf{NP}\xspace}  % bold notations
\newcommand{\cP}{\textbf{P}\xspace}
% \newcommand{\cNP}{\ensuremath{\mathcal{N\!P}}\xspace}  % round notations
% \newcommand{\cP}{\ensuremath{\mathcal{P}}\xspace}

% Machine states
\newcommand{\computing}{computing}
\newcommand{\idle}{idle}
\newcommand{\off}{of\!f}
\newcommand{\on}{on}
\newcommand{\ontooff}{\on\rightarrow\off}
\newcommand{\offtoon}{\off\rightarrow\on}

% CCGRID 2017
\newcommand{\ra}[1]{\renewcommand{\arraystretch}{#1}}

% Cluster 2017
\newcommand{\overbar}[1]{\mkern 1.5mu\overline{\mkern-1.5mu#1\mkern-1.5mu}\mkern 1.5mu}

\newlength\mylen % algorithm2e hack
\newcommand\myinput[1]{%
  \settowidth\mylen{\KwIn{}}%
  \setlength\hangindent{\mylen}%
  \hspace*{\mylen}#1\\}

\newcommand\myoutput[1]{% algorithm2e hack
  \settowidth\mylen{\KwOut{}}%
  \setlength\hangindent{\mylen}%
  \hspace*{\mylen}#1\\}



%%%%%%%%%%%%%%%%%%%%%%%
% Modular compilation %
%%%%%%%%%%%%%%%%%%%%%%%
\newif\ifwatermark
\watermarktrue

\newif\iftotalcompilation
\totalcompilationtrue

\newcommand{\inputchapter}[2]{%
  \ifdef{#1}%
    {\input{#2}\cleardoublepage}%
    {}%
}

% Modeling notations
\newcommand{\model}[2][]{\ensuremath{\mathcal{M}_{#1}\ifthenelse{\equal{#2}{}}{}{\!-\!}{#2}}\xspace}
\newcommand{\modelp}[2][]{\ensuremath{\mathcal{M'}_{#1}\!\ifthenelse{\equal{#2}{}}{}{\!-\!}{#2}}\xspace}
\newcommand{\noise}[2][]{\ensuremath{\mathcal{N}_{#1}\ifthenelse{\equal{#2}{}}{}{\!-\!}{#2}}\xspace}
\newcommand{\noisep}[2][]{\ensuremath{\mathcal{N'}_{#1}\!\ifthenelse{\equal{#2}{}}{}{\!-\!}{#2}}\xspace}
\newcommand{\norm}{\ensuremath{N}\xspace}
\newcommand{\mcdots}{\ensuremath{\!\cdot\!\cdot\!\cdot\!}\xspace}
\newcommand{\unif}[2]{\ensuremath{\mathcal{U}\left({#1},{#2}\right)}\xspace}

% Abreviations
\newcommand{\eg}{e.g.\@\xspace}
\newcommand{\ie}{i.e.\@\xspace}
\newcommand{\resp}{resp.\@\xspace}
\newcommand{\etal}{et~al.\@\xspace}


\definecolor{darkgray}{HTML}{333333}
\definecolor{gray}{HTML}{4D4D4D}
\definecolor{lightgray}{HTML}{BBBBBB}
\definecolor{green}{HTML}{C2E15F}
\definecolor{orange}{HTML}{FDA333}
\definecolor{purple}{HTML}{D3A4F9}
\definecolor{red}{HTML}{FB4485}
\definecolor{blue}{HTML}{6CE0F1}
\definecolor{pblue}{HTML}{0395DE}
\definecolor{materialpurple}{HTML}{9C27B0}
\definecolor{materialindigo}{HTML}{3F51B5}
\definecolor{materialblue}{HTML}{2196F3}
\definecolor{materialcyan}{HTML}{00BCD4}
\definecolor{materialteal}{HTML}{009688}
\definecolor{materialgreen}{HTML}{4CAF50}
\definecolor{materiallime}{HTML}{CDDC39}
\definecolor{materialamber}{HTML}{FFC107}
\definecolor{materialbrown}{HTML}{795548}
\definecolor{materialred}{HTML}{FF4436}
\definecolor{materialorange}{HTML}{FF5722}
\setbeamercolor{normal text}{bg=white}
\colorlet{mycolor}{materialblue}
\setbeamercolor{alerted text}{fg=mycolor}

\renewcommand{\emph}[1]{\SoulColor{lightorange}\hl{#1}}
\makeatletter
\let\HL\hl
\renewcommand\hl{%
  \let\set@color\beamerorig@set@color
  \let\reset@color\beamerorig@reset@color
  \HL}
\colorlet{mycolorlight}{mycolor!30}
\sethlcolor{mycolorlight}
\makeatother
\renewcommand<>{\hl}[1]{\only#2{\beameroriginal{\hl}}{#1}}
\renewcommand<>{\emph}[1]{\only#2{\beameroriginal{\hl}}{#1}}


\setlength{\interspacetitleruled}{0pt}%
\setlength{\algotitleheightrule}{0pt}%

% Using a small font for verbatim blocks
\def\changefont#1{%
  \setbeamertemplate{itemize/enumerate body begin}{#1}
  \setbeamertemplate{itemize/enumerate subbody begin}{#1}
  #1}
\makeatletter
\newcommand{\verbatimfont}[1]{\renewcommand{\verbatim@font}{\ttfamily#1}}
\makeatother
\verbatimfont{\scriptsize}%small
\let\endmintedbak=\endminted
\def\endminted{\endmintedbak\vspace{-1cm}}


\date{}
\author{Tom Cornebize}
\date{2 June 2021, PhD defense}
\title{High Performance Computing: Towards Better Performance Predictions and Experiments}

\begin{document}
\titlegraphic{
    \begin{picture}(0,0)
        \put(310,-200){\makebox(0,0)[rt]{
            \begin{minipage}{\textwidth}
                \includesvg[height=1cm]{img/logo_uga.svg}\hfill%
                \includegraphics[height=1cm]{img/slides/logo_cnrs.png}\hfill%
                \includegraphics[height=1cm]{img/slides/logo_inria.png}
            \end{minipage}
        }}
    \end{picture}
}

\maketitle

\begin{frame}{No science without computing}
    \begin{columns}
        \begin{column}[c]{.33\columnwidth}
            \includegraphics[width=\linewidth]{img/slides/arithmometer.png}
            Arithmomètre (1851)
        \end{column}
        \begin{column}[c]{.33\columnwidth}
            \includegraphics[width=\linewidth]{img/slides/eniac.jpg}
            ENIAC (1945)
        \end{column}
        \begin{column}[c]{.33\columnwidth}
            \includegraphics[width=\linewidth]{img/slides/fugaku.jpg}
            Fugaku (2021)
        \end{column}
    \end{columns}
    \vfill

    \pause

    Last decades:
    \begin{itemize}
        \item Exponential \alert{performance} improvements
        (\eg sequencing an entire human genome costed \SI{100000000}[\$]{} in 2001, \SI{1000}[\$]{} now)
        \item At the price of \alert{complexity} (both software and hardware)
    \end{itemize}
\end{frame}

\begin{frame}{Experimental study of computer performance}
    \begin{columns}
        \begin{column}[c]{.5\columnwidth}
            \includegraphics[width=\linewidth]{img/slides/cells.jpg}
        \end{column}
        \begin{column}[c]{.5\columnwidth}
            Similar to natural sciences
            \begin{flalign*}
                \text{Complexity} &\Rightarrow \text{Variability and Opacity}&\\
                                  &\Rightarrow \text{No perfect model}&\\
                                  &\Rightarrow \text{Need for \alert{experiments}}&\\
            \end{flalign*}
        \end{column}
    \end{columns}
    \pause
    \vfill
    Experiments can be carried in \alert{reality} or in \alert{simulation}
\end{frame}

\begin{frame}{Context}
    \textbf{Typical Performance Evaluation Questions} (Given my application and a supercomputer)

    \begin{minipage}[m]{0.35\linewidth}
        \includegraphics[width=\textwidth]{img/slides/computer_guy_meme.pdf}
    \end{minipage} %
    \begin{minipage}[m]{0.64\linewidth}
    \begin{itemize}
        \item \textbf{Before} running
        \begin{itemize}
            \item How many nodes?
            \item For how long?
            \item Which parameters?
        \end{itemize}
        \pause
        \item \textbf{After} running
        \begin{itemize}
            \item Performance as ``expected''?
            \item Problem in the app or the platform?
        \end{itemize}
    \end{itemize}
    \end{minipage}
    \pause

    \begin{center}
        So many large-scale runs, solely to tune performance?!?
    \end{center}

    \pause

    \textbf{Holy Grail: Predictive Simulation on a ``Laptop''}

    Capture the \textbf{whole application}  and \textbf{platform complexity}
\end{frame}

\begin{frame}[plain]
    \onslide<+->
    \begin{LARGE}
        Initial goal: \alert{\textbf{predict}} the performance of a parallel application
    \end{LARGE}
    \vfill
    \onslide<+->
    \begin{block}{Thesis contributions (towards this goal)}
        \begin{itemize}
            \item Case study: High Performance Linpack (HPL)
            \item Extensive (in)validation, comparing simulations with reality
            \item Modeling correctly the platform variability is key
        \end{itemize}
    \end{block}
    \onslide<+->
    \begin{block}{Thesis contributions (made on the way)}
        \begin{itemize}
            \item \textcolor<+->{lightgray}{Automation (of experiments, statistical analyzes, etc.)}
            \item \textcolor<.>{lightgray}{Experiment methodology, to bias or not to bias}
            \item Performance tests, to detect eventual platform changes
        \end{itemize}
    \end{block}
\end{frame}

\section{Performance prediction through simulation}%

\begin{frame}[fragile]{Sim(Em)ulation: The SMPI Approach}
    \begin{columns}
        \begin{column}[c]{.2\columnwidth}
            \scalebox{.8}{\begin{tikzpicture}[xscale=1,yscale=1]
                \node (nodehost) [name=nodehost]
                { \includegraphics[height=23mm]{img/slides/laptop.png}};

                %\node (nodelisting) [above right= -25mm of nodehost]%, overlay]
                % { \includegraphics[height=12mm]{fig/mpi-codelisting.png}};

                \node (nodeimagine) [
                    shape             = cloud callout,
                    cloud puffs       = 11,
                    aspect            = 1.5,
                    opacity           =.75,
                    draw              = black!90!white, % colour of the border
                    top color         = white,                % | filling of the node
                    bottom color      = black!30!white, % |
                    text              = black!90!white, % colour of the fonts
                    thick,                              % thickness of the border
                    above             = 5mm of nodehost,
                    minimum height    = 25mm,
                    minimum width     = 30mm,
                    callout pointer shorten=7mm,
                    callout absolute pointer={(285:5mm)},%(nodelisting.northwest)},%(285:5.5mm)},
                ]{};

                \node at (nodeimagine) {\includegraphics[width=15mm]{img/slides/cluster.png}};
            \end{tikzpicture}}
        \end{column}

        \begin{column}[c]{.85\columnwidth}
            \begin{picture}(0, 0)
                \put(167,-26){\hbox{
                    \includegraphics[width=2cm]{img/slides/simgrid_logo.pdf}
                }}
            \end{picture}
            \begin{block}{Full reimplementation of MPI on top of}% \alert{SimGrid}}
                \begin{itemize}
                    \item C/C++/F77/F90 codes run \alert{unmodified out of the box}
                    \item Simply replace mpicc/mpirun by smpicc/smpirun
                \end{itemize}
            \end{block}
            \pause
            \begin{block}{Emulation: how?}
                \begin{itemize}
                    \item Computations run for real on a laptop
                    \item Communications are faked, good fluid network models
                    \item \alert{Performance model} for the target platform
                \end{itemize}
            \end{block}
        \end{column}
    \end{columns}
    \pause
    \textbf{Contribution}: Skip the expensive computations (mostly \dgemm) and replace them by performance models
\end{frame}

\begin{frame}{Modeling Commputations}
    \begin{equation*}
        \texttt{dgemm}_{\uncover<3->{i}}(M,N,K) =
        \only<2>{\alpha.M.N.K}\uncover<3->{\underbrace{\alpha_i.M.N.K}_{\text{per host}}}
        \uncover<5->{+ \underbrace{\beta_i.M.N +  \gamma_i.N.K + \dots }_{\text{polynomial model}}}
        \uncover<6->{+ \underbrace{\norm(0,\alpha'_i.M.N.K + \dots)}_{\text{polynomial noise}}}
    \end{equation*}

    \begin{center}
        \begin{tabular}{p{0.45\textwidth}p{0.45\textwidth}}
            \uncover<3->{Different color $\Rightarrow$ different host} &
            \uncover<4->{For a particular host} \\

            \includegraphics<1>[width=\linewidth]{img/slides/dgemm_heterogeneity_calib_1.png} %
            \includegraphics<2>[width=\linewidth]{img/slides/dgemm_heterogeneity_calib_2.png} %
            \includegraphics<3->[width=\linewidth]{img/slides/dgemm_heterogeneity_calib_3.png} &

            \includegraphics<4>[width=\linewidth]{img/slides/dgemm_model_calib_1.png} %
            \includegraphics<5>[width=\linewidth]{img/slides/dgemm_model_calib_2.png} %
            \includegraphics<6->[width=\linewidth]{img/slides/dgemm_model_calib_3.png} \\

        \end{tabular}
    \end{center}
\end{frame}

\begin{frame}{Modeling Communications}
    \textbf{Hand-crafted non-blocking collective operations} intertwinned with computations

    \begin{minipage}[t][.7\textheight][t]{\textwidth}
        \begin{center}
            \includegraphics<2>[width=.9\linewidth]{img/prediction/modeling/network/mpi_calibration.png}
        \end{center}
    \end{minipage}
\end{frame}

\begin{frame}{Parenthesis: on the difficulties of experimentation}
    \alert{Experimental biases} when measuring \dgemm or MPI durations

    Effect on durations, but also other metrics (\eg CPU frequency)
    \onslide<+->
    \begin{itemize}[<+->]
        \item Sampling method for generating the sequence of calls
        \item Sample itself (for a given sampling method)
        \item Interferences between computations and communications
        \item Content of the matrices used by \dgemm
    \end{itemize}

    \onslide<+->
    Bias may be desirable in some situations
\end{frame}


\section{Validating the predictions}

\begin{frame}[fragile]{Internal behavior of the application}
    256 MPI ranks, interrupted after the $5^{\text{th}}$ iteration\medskip

    \newcommand{\ganttcaption}[2]{\rotatebox{90}{\hspace{.6cm}$\text{#1}\atop \text{#2}$\hspace{-1cm}}}%
    \begin{overlayarea}{\linewidth}{7cm}
        \begin{tabular}{c@{}c}
            \only<+->{\ganttcaption{\hspace{.3cm}Reality}{}          & \includegraphics[width=.93\linewidth]{img/prediction/validation/traces/gantt_reality.png} \\}%
            \only<+>{\ganttcaption{Simple kernel}{Simple network}    & \includegraphics[width=.93\linewidth]{img/prediction/validation/traces/gantt_simulation_deterministic-CPU_linear-DGEMM_deterministic-network.png}\\}%
            \only<+>{\ganttcaption{Simple kernel}{Complex network}   & \includegraphics[width=.93\linewidth]{img/prediction/validation/traces/gantt_simulation_deterministic-CPU_linear-DGEMM_stochastic-network.png}\\}%
            \only<+>{\ganttcaption{Complex kernel}{Complex network}  & \includegraphics[width=.93\linewidth]{img/prediction/validation/traces/gantt_simulation_stochastic-CPU_polynomial-DGEMM_stochastic-network.png} \\}%
        \end{tabular}
    \end{overlayarea}
\end{frame}


\begin{frame}[fragile]{Influence of the problem size}
    Now the complete run, with 1024 MPI ranks

    \tikzstyle{model_label}=[anchor=south west, font=\scriptsize]
    \begin{tikzpicture}
        \node[anchor=south west,inner sep=0] at (0,0){
            \begin{minipage}{\linewidth}
                \includegraphics<1>[width=0.9\linewidth]{img/slides/validation_performance_1.pdf}
                \includegraphics<2>[width=0.9\linewidth]{img/slides/validation_performance_2.pdf}
                \includegraphics<3>[width=0.9\linewidth]{img/slides/validation_performance_3.pdf}
                \includegraphics<4->[width=0.9\linewidth]{img/slides/validation_performance_4.pdf}
            \end{minipage}
        };
        \onslide<3->{\draw[-latex] (9.7, 4.5) to[bend left] node [midway, right, model_label, anchor=west]
            {\emph{\textbf{Heterogeneity}}} (9.7, 3.9) ;}
        \onslide<4->{\draw[-latex] (9.7, 3.9) to[bend left] node [midway, right, model_label,
            anchor=west]{\emph{\textbf{Variability}}} (9.7, 3.5) ;}
    \end{tikzpicture}
    \onslide<5->{
        \textbf{Take-Away Message}: \emph{accurate prediction}

        Modeling both \textbf{spatial} and \textbf{temporal} computation variability is essential
    }
\end{frame}

\begin{frame}[fragile]{Influence of a platform change}

    \tikzstyle{model_label}=[anchor=south west, font=\scriptsize]
    \begin{tikzpicture}
        \node[anchor=south west,inner sep=0] at (0,0){
            \begin{minipage}{\linewidth}
                \includegraphics<1>[width=0.9\linewidth]{img/slides/validation_temperature_1.pdf}
                \includegraphics<2>[width=0.9\linewidth]{img/slides/validation_temperature_2.pdf}
                \includegraphics<3->[width=0.9\linewidth]{img/slides/validation_temperature_3.pdf}
            \end{minipage}
        };
        \onslide<2->{\draw[-latex] (9.7, 4) to[bend left] node [midway, right, model_label, anchor=west]
            {\emph{\textbf{Overheating}}} (9.7, 3.5) ;}
    \end{tikzpicture}
    \onslide<2->{On four nodes, the cooling system malfunctionned for several weeks}
    \vfill
    \onslide<4->{
        \textbf{Take-Away Message}: Re-measuring \dgemm durations to generate a new model was enough to account for the
        platform change
    }
\end{frame}


\section{Performance tests}%

\begin{frame}{Regular measures}
    On a near-daily basis, run the \dgemm calibration code on
    \begin{picture}(0, 0)
        \put(-3, -11){\hbox{
            \includegraphics[width=2cm]{img/slides/grid5000-logo.pdf}
        }}
    \end{picture}

    454 nodes (792 CPU) from 12 clusters
    \pause

    For each CPU, collect:
    \begin{itemize}
        \item average \dgemm performance
        \item \dgemm coefficients of regression (\ie the model for simulation)
        \pause
        \item average CPU frequency
        \item average CPU power consumption
        \item average DRAM power consumption
        \item average temperature
    \end{itemize}
    \pause
    Each parameter is \alert{normally distributed} (thanks to CLT)
\end{frame}

\begin{frame}{Fluctuation interval}
    Given a sequence of \alert{old~observations} $x_1, \dots, x_n$ and a \alert{new~observation} $x_{n+1}$, how likely
    was it to observe $x_{n+1}$?

    \begin{minipage}{0.5\linewidth}
        \begin{center}
            \includegraphics[width=\linewidth]{img/slides/normal.pdf}
        \end{center}
    \end{minipage}\hfill%
    \begin{minipage}{0.48\linewidth}
        Take the sample mean \(\mu\) and standard deviation \(\sigma\) of the old observations

        \medbreak

        \(\mathbb{P}\left(x_{n+1}\in\left[\mu-2\sigma;\mu+2\sigma\right]\right) \approx 95\%\)
    \end{minipage}
\end{frame}

\begin{frame}{Fluctuation interval for several variables}
    With several variables, using their \alert{covariance~matrix}

    Example in dimension 2, with \(\mathbb{P}(x_{n+1} \in \text{interval}) \approx 99.5\%\)

    \begin{center}
        \includegraphics[width=0.8\linewidth]{img/experiment/non_regression/statistics/single_point.pdf}
    \end{center}
\end{frame}

\begin{frame}{Fluctuation interval for several measures}
    With several measures, using their \alert{average} and shrinking the interval

    Example with 5 measures (averages represented by crosses)

    \begin{center}
        \includegraphics[width=0.8\linewidth]{img/experiment/non_regression/statistics/several_points.pdf}
    \end{center}
\end{frame}

\begin{frame}{Result: performance fluctuation}
    \begin{center}
        \begin{minipage}[t][.4\textheight][t]{\textwidth}
            Performance fluctuation of the node dahu-14 \uncover<3>{(5-day window)}\\
            \includegraphics<1-2>[width=0.9\linewidth]{img/slides/evolution_dahu-14.pdf}
            \includegraphics<3>[width=0.9\linewidth]{img/slides/evolution_dahu-14_windowed.pdf}
        \end{minipage}
        \begin{minipage}[t][.4\textheight][t]{\textwidth}
            \uncover<2->{Performance fluctuation of the node dahu-32} \uncover<3>{(5-day window)}\\
            \includegraphics<2>[width=0.9\linewidth]{img/slides/evolution_dahu-32.pdf}
            \includegraphics<3>[width=0.9\linewidth]{img/slides/evolution_dahu-32_windowed.pdf}
        \end{minipage}
    \end{center}
\end{frame}

\begin{frame}{Result: performance overview}
    Overview of the performance on cluster dahu \uncover<2>{(5-day window)}\\
    \begin{center}
        \includegraphics<1>[width=0.9\linewidth]{img/slides/overview_dahu.pdf}
        \includegraphics<2>[width=0.9\linewidth]{img/slides/overview_windowed_dahu.pdf}
    \end{center}
\end{frame}


% See:
%   https://academia.stackexchange.com/a/88068/37904
%   https://unix.stackexchange.com/a/277987/106103
%   https://stackoverflow.com/a/5349842/4110059
%\begin{frame}[plain]
%    \centering
%    \setlength{\tabcolsep}{0pt}
%    \renewcommand{\arraystretch}{0}
%    \begin{tabular}{|c|c|c|}
%        \hline
%        \addframe{1}  & \addframe{6}  & \addframe{9}  \\
%        \hline
%        \addframe{11}  & \addframe{16}  & \addframe{24}  \\
%        \hline
%        \addframe{27}  & \addframe{32}  & \addframe{34}  \\
%        \hline
%        % Some ugly stuff, it would be great to find something better...
%        \vspace{10pt}\\
%        \multicolumn{3}{l}{\alert{\faicon{envelope}} \href{mailto:tom.cornebize@univ-grenoble-alpes.fr}{tom.cornebize@univ-grenoble-alpes.fr}}\\
%    \end{tabular}
%\end{frame}

\end{document}
