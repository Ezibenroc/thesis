%!TEX encoding = UTF-8 Unicode

% rubber: path ./sty
% rubber: path ./fig
% rubber: path ./img
% rubber: path ./inputs
% rubber: shell_escape

% The structure of this document is based on the my-thesis.tex example provided
% with the cleanthesis package (see http://cleanthesis.der-ric.de/).
% It also is inspired by the PhD manuscripts of Raphaël Bleuse, David Beniamine
% (see https://github.com/dbeniamine/Ph.D_thesis) and David Glesser.
% Uneeded packages and comments have been stripped out.

\pdfobjcompresslevel 0  % Produce PDF file respecting CINES requirements

\documentclass[%
    paper=A4,              % paper size
    twoside=true,          % onesite or twoside printing
    openright,             % doublepage cleaning ends up right side
    parskip=half,          % spacing value / method for paragraphs
    chapterprefix=true,    % prefix for chapter marks
    11pt,                  % font size
    headings=normal,       % size of headings
    bibliography=totoc,    % include bib in toc
    listof=totoc,          % include listof entries in toc
    titlepage=on,          % own page for each title page
    captions=tableabove,   % display table captions above the float env
    chapterprefix=false,   % do not display a prefix for chapters
    appendixprefix=false,  % do not display a prefix for appendix chapters
    draft=false,           % value for draft version
]{scrreprt}


% === PACKAGES IMPORT / CONFIGURATION =========================================
% --- ENCODING / FONTS -----------------

\usepackage[utf8]{inputenc}
\usepackage[T1]{fontenc}
\usepackage{textcomp}
\usepackage{lipsum}

% --- SPECIFIC ABBREVIATIONS MACROS -------------

% *** SPECIFIC MACROS ***
% specific macro definition for the paper
%
% requires amsthm, xspace

% theorem-like environment
\newtheorem{thm}{Theorem}
\newtheorem{lem}[thm]{Lemma}
\newtheorem{prop}{Proposition}
\newtheorem{propty}{Property}
\newtheorem{defn}{Definition}

% asymptotic complexity (Landau) notations
\newcommand{\landauO}{\ensuremath{\mathcal{O}}\xspace}
\newcommand{\landauOmega}{\ensuremath{\Omega}\xspace}
\newcommand{\landauTheta}{\ensuremath{\Theta}\xspace}
\newcommand{\landauo}{\ensuremath{o}\xspace}
\newcommand{\landauomega}{\ensuremath{\omega}\xspace}
\newcommand{\landauorder}{\ensuremath{\sim}\xspace}

% scheduling notations
\newcommand{\graham}[3]{\mbox{\ensuremath{#1\mid#2\mid#3}}}
\newcommand{\Cmax}{\ensuremath{C_{\max}}\xspace}

% complexity classes
\newcommand{\cNP}{\textbf{NP}\xspace}  % bold notations
\newcommand{\cP}{\textbf{P}\xspace}
% \newcommand{\cNP}{\ensuremath{\mathcal{N\!P}}\xspace}  % round notations
% \newcommand{\cP}{\ensuremath{\mathcal{P}}\xspace}

% Machine states
\newcommand{\computing}{computing}
\newcommand{\idle}{idle}
\newcommand{\off}{of\!f}
\newcommand{\on}{on}
\newcommand{\ontooff}{\on\rightarrow\off}
\newcommand{\offtoon}{\off\rightarrow\on}

% CCGRID 2017
\newcommand{\ra}[1]{\renewcommand{\arraystretch}{#1}}

% Cluster 2017
\newcommand{\overbar}[1]{\mkern 1.5mu\overline{\mkern-1.5mu#1\mkern-1.5mu}\mkern 1.5mu}

\newlength\mylen % algorithm2e hack
\newcommand\myinput[1]{%
  \settowidth\mylen{\KwIn{}}%
  \setlength\hangindent{\mylen}%
  \hspace*{\mylen}#1\\}

\newcommand\myoutput[1]{% algorithm2e hack
  \settowidth\mylen{\KwOut{}}%
  \setlength\hangindent{\mylen}%
  \hspace*{\mylen}#1\\}



%%%%%%%%%%%%%%%%%%%%%%%
% Modular compilation %
%%%%%%%%%%%%%%%%%%%%%%%
\newif\ifwatermark
\watermarktrue

\newif\iftotalcompilation
\totalcompilationtrue

\newcommand{\inputchapter}[2]{%
  \ifdef{#1}%
    {\input{#2}\cleardoublepage}%
    {}%
}

% Modeling notations
\newcommand{\model}[2][]{\ensuremath{\bm{M}_{#1}\ifthenelse{\equal{#2}{}}{}{\!-\!}{#2}}\xspace}
\newcommand{\modelp}[2][]{\ensuremath{\bm{M'}_{#1}\!\ifthenelse{\equal{#2}{}}{}{\!-\!}{#2}}\xspace}
\newcommand{\noise}[2][]{\ensuremath{\bm{N}_{#1}\ifthenelse{\equal{#2}{}}{}{\!-\!}{#2}}\xspace}
\newcommand{\noisep}[2][]{\ensuremath{\bm{N'}_{#1}\!\ifthenelse{\equal{#2}{}}{}{\!-\!}{#2}}\xspace}
\newcommand{\norm}{\ensuremath{\mathcal{N}}\xspace}
\newcommand{\mcdots}{\ensuremath{\!\cdot\!\cdot\!\cdot\!}\xspace}
\newcommand{\unif}[2]{\ensuremath{\mathcal{U}\left({#1},{#2}\right)}\xspace}

% Abreviations
\newcommand{\eg}{e.g.\@\xspace}
\newcommand{\ie}{i.e.\@\xspace}
\newcommand{\aka}{a.k.a.\@\xspace}
\newcommand{\resp}{resp.\@\xspace}
\newcommand{\etal}{et~al.\@\xspace}
\newcommand{\dgemm}{\texttt{dgemm}\@\xspace}
\newcommand{\recv}{\texttt{MPI\_Recv}\@\xspace}
\newcommand{\send}{\texttt{MPI\_Send}\@\xspace}
\newcommand{\isend}{\texttt{MPI\_Isend}\@\xspace}
\newcommand{\iprobe}{\texttt{MPI\_Iprobe}\@\xspace}
\newcommand{\pyce}{\texttt{pycewise}\@\xspace}

% Referencing several times a footnote
% See https://tex.stackexchange.com/a/54240
\newcommand{\savefootnote}[2]{\footnote{\label{#1}#2}}
\newcommand{\repeatfootnote}[1]{\textsuperscript{\ref{#1}}}

\input{macros.include.tex}


\ifwatermark
% https://texblog.org/2012/02/17/watermarks-draft-review-approved-confidential/
\usepackage{draftwatermark}
\usepackage{datetime}
 \SetWatermarkColor[rgb]{0.8,0.9,1}
 \SetWatermarkText{%
    \begin{minipage}{6in}
    \begin{center}
      \settimeformat{hhmmsstime}
      {\fontsize{24}{12}\selectfont DRAFT\\\today, \currenttime}
    \end{center}
    \end{minipage}
  }
\SetWatermarkScale{1}

\fi

% \usepackage{kpfonts}

% --- i18n, l10n -----------------------

\usepackage[french,english]{babel}  % load default language last
\usepackage[french,english]{isodate}  % load default language last

% --- PAGE SETTING ---------------------

\usepackage[%
    figuresep=colon,          % label separator for captions
    hangfigurecaption=false,  % use hanging figure label (within margin)
    hangsection=true,         % use hanging section label (within margin)
    hangsubsection=true,      % use hanging subsection label (within margin)
    colorize=full,            % define color mode
    colortheme=bluemagenta,   % what colors to use
    bibsys=biber,             % citation manamgement engine
    bibfile=references,       % bibtex file
    bibstyle=alphabetic,      % reference style
]{cleanthesis}

% --- GRAPHICS / FIGURES ---------------

\usepackage{epsfig}  % XXX: to be removed when CCPE figures are out

\newcommand{\SetFigFont}[3]{\fontsize{#1}{#2pt}\normalfont\sffamily}  % XXX: to be removed when CCPE figures are out

\usepackage{tikz}
\usepackage{pgfplots}
\pgfplotsset{compat=1.13}
\usetikzlibrary{arrows,shapes,positioning,shadows,trees,calc,decorations.text}

\usepackage{bm}

\usepackage{titlesec}

% --- MATHEMATICS TYPESETTING ----------

\usepackage{amsmath}
\usepackage{amssymb}
\usepackage{amsthm}

% --- MODULAR CHAPTER COMPILATION -----

\usepackage{etoolbox}

% --- MISC. PACKAGES -------------------

\usepackage[a4paper]{uga}  % UGA title page style (see titlepage.tex)
\usepackage{xspace}  % easy macros definition
\usepackage{booktabs}
%\usepackage[font=footnotesize]{subfig}
\usepackage{subfigure}
\usepackage[ruled,noend]{algorithm2e}
\usepackage{csquotes}
\usepackage{tabularx}
\usepackage{pifont}
\newcommand{\cmark}{\ding{51}}%
\newcommand{\xmark}{\ding{55}}%
\usepackage{pdflscape}

\usepackage[binary-units,group-digits,group-separator={,}]{siunitx}
% Uncomment to have stuff like 6.7e-11, keep it commented for 6.7\times10^{-11}
%\sisetup{output-exponent-marker=\ensuremath{\mathrm{e}}}
\DeclareSIUnit\flop{Flop}
\DeclareSIUnit\flops{\flop\per\second}
\newcommand{\Num}[1]{\num[group-separator={,}]{#1}\xspace}
\newcommand{\NSI}[2]{\SI[group-separator={,}]{#1}{#2}\xspace}
\DeclareSIUnit\core{core}

\usepackage{color,colortbl}
\definecolor{gray98}{rgb}{0.98,0.98,0.98}
\definecolor{gray20}{rgb}{0.20,0.20,0.20}
\definecolor{gray25}{rgb}{0.25,0.25,0.25}
\definecolor{gray16}{rgb}{0.161,0.161,0.161}
\definecolor{gray80}{rgb}{0.8,0.8,0.8}
\definecolor{gray60}{rgb}{0.6,0.6,0.6}
\definecolor{gray30}{rgb}{0.3,0.3,0.3}
\definecolor{bgray}{RGB}{248, 248, 248}
\definecolor{amgreen}{RGB}{77, 175, 74}
\definecolor{amblu}{RGB}{55, 126, 184}
\definecolor{amred}{RGB}{228,26,28}
\definecolor{amdove}{RGB}{102,102,122}
\usepackage{xcolor}
\definecolor{myblue}{cmyk}{1, .50, .10, .01}  % see definition of "bluemagenta" in cleanthesis.sty
\usepackage[procnames]{listings}
\lstset{ %
    backgroundcolor=\color{gray98},    % choose the background color; you must add \usepackage{color} or \usepackage{xcolor}
    basicstyle=\ttfamily\scriptsize,        % the size of the fonts that are used for the code
    breakatwhitespace=false,          % sets if automatic breaks should only happen at whitespace
    breaklines=true,                  % sets automatic line breaking
    showlines=true,                   % sets automatic line breaking
    captionpos=b,                     % sets the caption-position to bottom
    commentstyle=\color{gray60},      % comment style
    extendedchars=true,               % lets you use non-ASCII characters; for 8-bits encodings only, does not work with UTF-8
    frame=single,                     % adds a frame around the code
    keepspaces=true,                  % keeps spaces in text, useful for keeping indentation of code (possibly needs columns=flexible)
    columns=flexible,
    keywordstyle=\color{amblu},       % keyword style
    procnamestyle=\color{colorfuncall},      % procedures style
    language=[95]fortran,             % the language of the code
    numbers=left,                     % where to put the line-numbers; possible values are (none, left, right)
    numbersep=5pt,                    % how far the line-numbers are from the code
    numberstyle=\tiny\color{gray20},  % the style that is used for the line-numbers
    rulecolor=\color{gray20},         % if not set, the frame-color may be changed on line-breaks within not-black text (\eg comments (green here))
    showspaces=false,                 % show spaces everywhere adding particular underscores; it overrides 'showstringspaces'
    showstringspaces=false,           % underline spaces within strings only
    showtabs=false,                   % show tabs within strings adding particular underscores
    stepnumber=2,                     % the step between two line-numbers. If it's 1, each line will be numbered
    stringstyle=\color{amdove},       % string literal style
    tabsize=2,                        % sets default tabsize to 2 spaces
    % title=\lstname,                    % show the filename of files included with \lstinputlisting; also try caption instead of title
    procnamekeys={call}
}
\definecolor{colorfuncall}{rgb}{0.6,0,0}

% --- PDF SUPPORT ----------------------

\usepackage{bookmark,hyperref}  % import as last package

%% Citing software
\ExecuteBibliographyOptions{
    halid=true,
    swhid=true,
    swlabels=true,
    vcs=true,
    license=true
}

\hypersetup{
    unicode=true,
    plainpages=false,
    colorlinks=false,
    pdfborder={0 0 0},
    breaklinks=true,  % allow line breaks inside links
    bookmarksnumbered=true,
    bookmarksopen=true,
}

% --- Custom TODO line ------------------------
\usepackage{mdframed}
\newmdenv[%
  linecolor=red,%
  backgroundcolor=red!20,%
  linewidth=3pt,%
  hidealllines=true,%
  leftline=true,%
]{todoenv}
\newcommand{\todo}[1]{\begin{todoenv}#1\end{todoenv}}
\newcommand{\todoC}[1]{\todo{#1}}


% === MACROS DEFINITION =======================================================

% --- COMMON ABBREVIATIONS MACROS -------------

% latin abbreviations, see:
%   - http://www.sussex.ac.uk/informatics/punctuation/capsandabbr/abbr
% comment by Sascha Hunold, see also:
%   - https://www.ieee.org/documents/style_manual.pdf
%     (p. 32, Short Reference List of Italics)
%   - http://web.ece.ucdavis.edu/~jowens/commonerrors.html

% === META DATA ===============================================================

%\title{Experiment Driven Performance Modeling for High Performance Computing}
%\title{Measuring, Modeling, Predicting and Testing Performance of Supercomputers}
%\title{High Performance Computing: a Contribution for better Experiments and Performance Predictions}
\title{High Performance Computing: Towards Better Performance Predictions and Experiments}
\author{Tom \textsc{Cornebize}}

% fill pdf meta data
\makeatletter
\hypersetup{pdftitle=\@author's thesis}
\hypersetup{pdfsubject=\@title}
\hypersetup{pdfauthor=\@author}
\makeatother

\usepackage{enumitem}

% === DOCUMENT CONTENT ========================================================

\begin{document}

% --- FRONT MATTER ---------------------

\iftotalcompilation
\pdfbookmark[-1]{Front Matter}{Front Matter}

\pagenumbering{gobble}  % do not count title page
\pagestyle{empty}  % no header nor footer

%!TEX encoding = UTF-8 Unicode

\begin{titlepage}
%
\pdfbookmark[0]{Cover}{Cover}
%
\begin{otherlanguage}{french}

% UGA meta data ------------------------

% \Universite{}
% \Grade{}
\Specialite{Informatique}
\Arrete{\printdate{2016-05-25}}
\Directeur{Arnaud \textsc{Legrand}, CNRS}
%\CoDirecteur{Your\textsc{Co-Director}}
\Laboratoire{Laboratoire d'Informatique de Grenoble}
\EcoleDoctorale{MSTII}
\Depot{\printdate{1970-01-01}} % XXXXXXXXXXXXXXXXXXXXXXXXXXXXXXXXXXXXXX

\makeatletter
\Auteur{{\@author}}
\Titre{Calcul haute performance : vers de meilleures prédictions de performance et expériences}
\Soustitre{\foreignlanguage{english}{\@title}}
\makeatother

% UGA meta data: jury ------------------

\Jury{
    % \UGTPresidente{}{}

    \UGTRapporteur{Patrick \textsc{Carribault}}
                  {Ingénieur-Chercheur,
                   CEA, France}

    \UGTRapporteur{Henri \textsc{Casanova}}
                  {Professeur,
                   University of Hawai`i at Mānoa,
                   États-Unis}

    \UGTExaminateur{Abhinav \textsc{Bhatele}}
                   {Professeur associé,
                   University of Maryland, États-Unis}

    \UGTExaminatrice{Camille \textsc{Coti}}
                   {Professeure associée,
                   Université Sorbonne Paris Nord, France}

    \UGTExaminatrice{Amina \textsc{Guermouche}}
                   {Professeure associée,
                   Télécom SudParis, France}

    \UGTExaminateur{Jean-François \textsc{Méhaut}}
                   {Professeur,
                   Université Grenoble Alpes, France}

    \UGTExaminateur{Christian \textsc{Plessl}}
                   {Professeur,
                   Universität Paderborn, Allemagne}

    % \UGTCoEncadrant{Giorgio \textsc{Lucarelli}}
    %               {Maître de conférences,
    %               LCOMS, Université de Lorraine, Metz, France}

%    \UGTInvite{Sascha \textsc{Hunold}}
%              {Professeur associé,
%              Technische Universität Wien, Autriche}

%    \UGTInvite{Stéphane \textsc{Pralet}}
%              {Ingénieur-Chercheur,
%              Atos, France}

    \UGTDirecteur{Arnaud \textsc{Legrand}}
                {Directeur de recherche,
                CNRS, France}
}

\Sethpageshift{0pt}  % adjust horizontal position
\Setvpageshift{100pt}  % adjust vertical position
\MakeUGthesePDG  % actually generate the title page

\end{otherlanguage}
\end{titlepage}
 %%%%%%%%%%%%%%%%%%%%%%%%%%%%%%%%%%%%%%%%%%%%%%%%%%%%%%%%%%%%%%%%%%  COMMENTED OUT FOR SEPARATE CHAPTER COMPILATIONS
\cleardoublepage %%%%%%%%%%%%%%%%%%%%%%%%%%%%%%%%%%%%%%%%%%%%%%%%%%%%%%%%%%%%%%%%%%%%%%%  COMMENTED OUT FOR SEPARATE CHAPTER COMPILATIONS

\pagenumbering{roman}  % roman page numbering (e.g., i ii iii), reset counter

%%!TEX encoding = UTF-8 Unicode

\pdfbookmark[0]{Dedication}{Dedication}
%
\null\vspace{\stretch{1}}
%
\begin{thesis_quotation}
%
\begin{flushright}

I dedicate this thesis to my grumpy cat.

\end{flushright}
%
\end{thesis_quotation}
%
\vspace{\stretch{2}}\null
 %%%%%%%%%%%%%%%%%%%%%%%%%%%%%%%%%%%%%%%%%%%%%%%%%%%%%%%%%%%%%%%%%  COMMENTED OUT FOR SEPARATE CHAPTER COMPILATIONS
%\cleardoublepage %%%%%%%%%%%%%%%%%%%%%%%%%%%%%%%%%%%%%%%%%%%%%%%%%%%%%%%%%%%%%%%%%%%%%%%  COMMENTED OUT FOR SEPARATE CHAPTER COMPILATIONS
%
%%!TEX encoding = UTF-8 Unicode

\pdfbookmark[0]{Epigraph}{Epigraph}
%
\null\vspace{\stretch{1}}
%
\begin{otherlanguage}{french}
%
\cleanchapterquote%
{
    Elle est o\`{u} la poulette ?
}%
{
	Kadoc \textsc{De Vannes}
}%
{
}
%
\end{otherlanguage}
%
\vspace{\stretch{2}}\null
 %%%%%%%%%%%%%%%%%%%%%%%%%%%%%%%%%%%%%%%%%%%%%%%%%%%%%%%%%%%%%%%%%%%  COMMENTED OUT FOR SEPARATE CHAPTER COMPILATIONS
%\cleardoublepage %%%%%%%%%%%%%%%%%%%%%%%%%%%%%%%%%%%%%%%%%%%%%%%%%%%%%%%%%%%%%%%%%%%%%%%  COMMENTED OUT FOR SEPARATE CHAPTER COMPILATIONS

\pagestyle{plain}  % display page number only

%!TEX encoding = UTF-8 Unicode

\addchap{Acknowledgments}

Experiments presented in this paper were carried out using the \mbox{Grid'5000}
testbed, supported by a scientific interest group hosted by \mbox{Inria} and including
\mbox{CNRS}, \mbox{RENATER} and several Universities as well as other organizations.

I would like to thank all the authors who upload their papers on open archives like HAL or Arxiv. Likewise, thanks a lot
to Alexandra \textsc{Elbakyan} for the creation of Sci-Hub, which helped me a lot for retrieving articles behind
paywalls.

Thank you to the members of the jury for kindly accepting to be part of this committee. In particular, thanks a lot to
the two reviewers, Patrick \textsc{Carribault} and Henri \textsc{Casanova}, who made excellent comments on the
manuscript which helped to improve the final version.

Thank you to all the Simgrid contributors for creating and maintaining the software on which is based a great part of
this thesis. They have been very helpful and friendly in numerous occasions.

Thank you to the Grid'5000 staff, I cannot emphasize enough how much their work is important. This platform is a
wonderful tool for computer science experiments.

Thanks a lot to the members of Polamove and Dataris teams who welcomed me for these last years. The days always pass
extremely fast with (long) coffee breaks. I am waiting eagerly the end of this pandemic to resume the traditional summer
barbecue.

A huge thank you to Arnaud, my advisor. I learned a lot during this thesis and overcomed my dislike for probabilities
and statistics. Above anything else, thank you for your great kindness and empathy.  Back in late 2016, when I was
looking for a master internship, I read the advice that the advisor should be the most important criteria when planning
a PhD, even more so than the thesis subject or the host institution. I followed this advice and never regretted.

Finally, I would like to thank my familly, who helped me and supported me a lot. In particular, Maïlys, my wife, with
whome I am extremely fortunate and very happy to share my life, and Yaël, my baby, still too young to understand this
text, but already a wonderful person.
 %%%%%%%%%%%%%%%%%%%%%%%%%%%%%%%%%%%%%%%%%%%%%%%%%%%%%%%%%%%%  COMMENTED OUT FOR SEPARATE CHAPTER COMPILATIONS
\cleardoublepage %%%%%%%%%%%%%%%%%%%%%%%%%%%%%%%%%%%%%%%%%%%%%%%%%%%%%%%%%%%%%%%%%%%%%%%  COMMENTED OUT FOR SEPARATE CHAPTER COMPILATIONS

\addchap{Abstract / \foreignlanguage{french}{Résumé}}

\section*{Abstract}

The English abstract.


\clearpage
% -----------------------------------------------------------------------------

\begin{otherlanguage}{french}

\section*{Résumé}

Le résumé en français.


\end{otherlanguage}
 %%%%%%%%%%%%%%%%%%%%%%%%%%%%%%%%%%%%%%%%%%%%%%%%%%%%%%%%%%%%%  COMMENTED OUT FOR SEPARATE CHAPTER COMPILATIONS
\cleardoublepage %%%%%%%%%%%%%%%%%%%%%%%%%%%%%%%%%%%%%%%%%%%%%%%%%%%%%%%%%%%%%%%%%%%%%%%  COMMENTED OUT FOR SEPARATE CHAPTER COMPILATIONS

\setcounter{tocdepth}{2}  % define depth of ToC
\phantomsection\addcontentsline{toc}{chapter}{{\contentsname}}  % display ToC to ToC
\tableofcontents %%%%%%%%%%%%%%%%%%%%%%%%%%%%%%%%%%%%%%%%%%%%%%%%%%%%%%%%%%%%%%%%%%%%%%%  COMMENTED OUT FOR SEPARATE CHAPTER COMPILATIONS
\cleardoublepage %%%%%%%%%%%%%%%%%%%%%%%%%%%%%%%%%%%%%%%%%%%%%%%%%%%%%%%%%%%%%%%%%%%%%%%  COMMENTED OUT FOR SEPARATE CHAPTER COMPILATIONS
\fi

% --- MAIN MATTER ----------------------

\pagenumbering{arabic}  % arabic page numbering (e.g., 1 2 3), reset counter
\pagestyle{scrheadings}  % display header and footer
\addtocontents{toc}{\bigskip}  % visual hint for numbering change in ToC



% main chapters go here
\bookmarksetup{startatroot}
\inputchapter{\includechapterintroduction}{chapter_introduction.tex}
\inputchapter{\includechapterprediction}{chapter_prediction.tex}
\inputchapter{\includechapterexperiment}{chapter_experiment.tex}

\bookmarksetup{startatroot}  % https://stackoverflow.com/questions/1483396/how-to-explicitly-end-a-part-in-latex-with-hyperref
\addtocontents{toc}{\bigskip}
\inputchapter{\includechapterconclusion}{chapter_conclusion.tex}

% --- BACK MATTER ----------------------

\iftotalcompilation
\pdfbookmark[-1]{Back Matter}{Back Matter}

\pagenumbering{arabic}  % arabic page numbering (e.g., 1 2 3), reset counter
\renewcommand*{\thepage}{A\arabic{page}}  % prepend A to appendix page number
\pagestyle{scrheadings}  % display header and footer
\addtocontents{toc}{\bigskip}  % visual hint for numbering change in ToC
\fi

\appendix
\inputchapter{\includechapterappendix}{chapter_appendix.tex}

\iftotalcompilation
\cleardoublepage %%%%%%%%%%%%%%%%%%%%%%%%%%%%%%%%%%%%%%%%%%%%%%%%%%%%%%%%%%%%%%%%%%%%%%%  COMMENTED OUT FOR SEPARATE CHAPTER COMPILATIONS

{
    \setstretch{1.1}
    \renewcommand{\bibfont}{\normalfont\small}
    \setlength{\biblabelsep}{0pt}
    \setlength{\bibitemsep}{0.5\baselineskip plus 0.5\baselineskip}
    \printbibliography
}

%{\listoffigures \let\cleardoublepage\  \listoftables} %%%%%%%%%%%%%%%%%%%%%%%%%%%%%%%%%%  COMMENTED OUT FOR SEPARATE CHAPTER COMPILATIONS

\cleardoublepage %%%%%%%%%%%%%%%%%%%%%%%%%%%%%%%%%%%%%%%%%%%%%%%%%%%%%%%%%%%%%%%%%%%%%%%  COMMENTED OUT FOR SEPARATE CHAPTER COMPILATIONS



 %%%%%%%%%%%%%%%%%%%%%%%%%%%%%%%%%%%%%%%%%% List of publications %%%%%%%%%%%%%%%%%
% {
%     \clearpage
%     Additionally, the work conducted in this dissertation directly led to the
%     following communications.

%     \begin{refsection}
%     \makeatletter\@beginparpenalty=10000\makeatother  % prevent page break before list
%     \defbibenvironment{itembib}{\itemize}{\enditemize}{\item}
%     \nocite{*}
%     %
%     \paragraph{Peer-reviewed international conferences}
%     \printbibliography[heading=none,env=itembib,keyword={own},keyword={conference}]
%     %
%     \paragraph{Peer-reviewed international workshops}
%     \printbibliography[heading=none,env=itembib,keyword={own},keyword={workshop}]
%     \paragraph{National workshops}
%     \printbibliography[heading=none,env=itembib,keyword={own},keyword={nworkshop}]
%     \end{refsection}
% }
%%%%%%%%%%%%%%%%%%%%%%%%%%%%%%%%%%%%%%%%%% List of publications %%%%%%%%%%%%%%%%%


\cleardoublepage %%%%%%%%%%%%%%%%%%%%%%%%%%%%%%%%%%%%%%%%%%%%%%%%%%%%%%%%%%%%%%%%%%%%%%%  COMMENTED OUT FOR SEPARATE CHAPTER COMPILATIONS

% --- BACK COVER -----------------------

\pagenumbering{gobble}  % do not count any more
\pagestyle{empty}  % no header nor footer

\cleardoubleevenpage  % ensure even page for back cover %%%%%%%%%%%%%%%%%%%%%%%%%%%%%%%%%  COMMENTED OUT FOR SEPARATE CHAPTER COMPILATIONS
\areaset[0pt]{\paperwidth}{\paperheight}  % hack to reset margins
%
\newlength{\bcmargin}\setlength{\bcmargin}{1.4cm}  % back cover margin length
%
\centering
%
\null\vspace*{\dimexpr\bcmargin-\headsep\relax}
%
\begin{minipage}{\dimexpr\paperwidth-\bcmargin-\bcmargin\relax}
%
\pdfbookmark[0]{Back Cover}{Back Cover}
%
\noindent{\usekomafont{section}Abstract}\par
%
The English abstract.

%
\vspace{3ex}\hrule\vspace{2ex}
%
\begin{otherlanguage}{french}
%
\noindent{\usekomafont{section}Résumé}\par
%
Le résumé en français.

%
\end{otherlanguage}
%
\end{minipage}
 %%%%%%%%%%%%%%%%%%%%%%%%%%%%%%%%%%%%%%%%%%%%%%%%%%%%%%%%%%%%%%%%%%%%%%%  COMMENTED OUT FOR SEPARATE CHAPTER COMPILATIONS
\fi

% === END OF DOCUMENT =========================================================

\end{document}
